\documentclass[aps,prl,twocolumn]{revtex4-1}

\usepackage{graphicx}
\graphicspath{{figures/}}
\usepackage{amsmath}
\usepackage{SIunits}

\renewcommand{\vec}[1]{\mathbf{#1}}
\newcommand{\uvec}[1]{\hat{#1}}
\newcommand{\abs}[1]{\left\vert #1 \right\vert}
\newcommand{\real}[1]{\Re\left\{ #1 \right\}}
\newcommand{\imaginary}[1]{\Im\left\{ #1 \right\}}
\newcommand{\mytilde}{\raise.17ex\hbox{$\scriptstyle\mathtt{\sim}$}}
\newcommand{\pos}{\left(\mathbf{r} \right)}
\newcommand{\bigM}{\mathbf{M}_{mn}}
\newcommand{\bigN}{\mathbf{N}_{mn}}
%\newcommand{\pol}{\mathbf{\rh}}

\begin{document} 

\title{Force on a spherical particle in an optical conveyor}

\author{David Ruffner}
\author{David G. Grier}
\affiliation{Department of Physics and Center for Soft Matter Research,
  New York University, New York, NY 10003}

\date{\today}
\begin{abstract}
  Derivation of the calculation of the force on a spherical particle
in an optical conveyor using Lorenz-Mie theory. 




\end{abstract}



\maketitle 

In order to calculate the force we must write down the incident field 
in terms of the vector spherical harmonics. This allows us to calculate
the scattered field by matching the boundary conditions at the scattering
sphere
\cite{barton_internal_1988}. Furthermore given the scattered field we
can calculate the force and torque on the spherical particle 
\cite{barton_theoretical_1989}. 

The field of an optical conveyor is
a sum of two Bessel beams, and each Bessel beam can be thought of as
a cone of plan waves \cite{cizmar_sub-micron_2006}. The electric
field of the Bessel beam as a function of position, $\mathbf{r}$, 
can be written as,
\begin{equation}
  \label{eq:besselfield}
  \mathbf{E}\pos =  E_0 \int_0^{2\pi} \mathbf{e}_0(\theta_0,\phi^\prime) 
      e^{i \mathbf{k}(\theta_0,\phi^\prime) \cdot \mathbf{r} } d\phi^\prime,
\end{equation}
Where $E_0$ is the amplitude of each component plane wave, 
$\mathbf{e}_0(\theta_0,\phi^\prime)$ is the polarization, and 
$\mathbf{k}(\theta_0,\phi^\prime)$ is the wavevector. The cone angle, $\theta_0$,
gives us the wavevector of each component plane wave,
\begin{align}
  \label{eq:wavevector}
  \mathbf{k}(\theta_0,\phi^\prime) &= k \hat{r}(\theta_0,\phi^\prime) \\
  &= k(\sin\theta_0 \cos \phi^\prime \hat{x} +
       \sin\theta_0 \sin \phi^\prime \hat{y} +
       \cos\theta_0) 
\end{align}
 in terms of the wavenumber, $k = 2 \pi n_{ext}/\lambda$, where $n_{ext}$ is the
refractive index of the medium and $\lambda$ is the wavelength of the 
light. If we assume that the Bessel beam was formed from light polarized 
in the $\hat{x}$ direction, then we can get the polarization of 
each plane wave component decomposing it in terms of $\hat{\theta}$
and $\hat{\phi^\prime}$ since these are the two directions transverse to 
$\mathbf{k}(\theta_0,\phi^\prime)$. Consequently the polarization of 
each plane wave component becomes,
\begin{equation}
  \label{eq:polarization}
  \mathbf{e}_0(\theta_0,\phi^\prime) = \cos \phi^\prime \,\hat{\theta}(\theta_0,\phi^\prime)
                               +\sin \phi^\prime\, \hat{\phi^\prime}(\theta_0,\phi^\prime).
\end{equation}

The electric field can always expressed in terms of the
vector spherical harmonics $\bigM^{(1)}$ and $\bigN^{(1)}$ 
\cite{bohren_absorption_2008}, with the following expansion,
\begin{equation}
  \label{eq:expansion}
  \mathbf{E} = -i \sqrt{\frac{2n+1}{4\pi}\frac{(n-m)!}{(n+m)!}}
     \sum_{n=1}^{\infty}\sum_{m=-n}^{n} \left(
     p_{mn} \bigN^{(1)}+q_{mn} \bigM^{(1)} \right),
\end{equation}
Where we are neglecting the position dependence for clarity.
The coefficients $p_{mn}$ and $q_{mn}$ are called the beam shape coefficients
(BSCs) and they can be determined by the following integral,
\begin{equation}
  \label{eq:pmnintegral}
  p_{mn} = i \sqrt{\frac{4\pi}{2n+1}\frac{(n+m)!}{(n-m)!}} 
          \frac{\int_{0}^{2\pi}\int_{0}^{\pi} 
             \mathbf{E} \cdot \bigN^* 
             \sin \theta d\theta d\phi^\prime}
          {\int_{0}^{2\pi}\int_{0}^{\pi} 
             |\bigN|^2
             \sin \theta d\theta d\phi^\prime},
\end{equation}
And a similar one for $q_{mn}$ with $\bigM$ instead of $\bigN$.
 This integral is difficult in general
however Taylor and Love solved it analytically in the case of a 
Bessel beam \cite{taylor_multipole_2009-1}. Essentially they wrote
the Bessel beam in terms of a cone of plane waves like we have above.
The found the BSCs for each plane wave as was shown in 
\cite{mackowski_calculation_1994}, and then added them all together and
were able to compute the integral. This enables us to write down the
BSCs for a Bessel beam,
\begin{align}
  \label{eq:besselBSCs}
  p_{mn}\pos = &E_0 U_n e^{ikz \cos \theta_0} \\
          &\, [ \tilde{\tau}_{mn}(\cos\theta_0)\, I^+(\rho,\phi) + \\
          &\,\,\, \tilde{\pi}_{mn}(\cos\theta_0)\, I^-(\rho,\phi) ]
\end{align}
where $\rho = \sqrt{x^2+y^2}$ is the distance from the beam axis, and 
$\phi = \arctan(-y/x)- \pi/2$ is the azimuthal angle. To calculate
$q_{mn}$ just exchange the $\tilde{\tau}_{mn}$ with the $\tilde{\pi}_{mn}$.
There is the factor
$U_n = \frac{4 \pi i^n}{n(n+1)}$ and then functions 
in terms of the modified Legendre Polynomials, $P^m_n$,
\begin{align}
  \label{eq:anglefunctions}
  \tilde{\pi}_{mn}(\cos\theta_0) &= 
      \sqrt{\frac{2n+1}{4\pi}\frac{(n-m)!}{(n+m)!}}
      \frac{m}{\sin\theta_0}P^m_n(\cos\theta_0),\\
  \tilde{\tau}_{mn}(\cos\theta_0) &= 
      \sqrt{\frac{2n+1}{4\pi}\frac{(n-m)!}{(n+m)!}}
      \frac{d}{d\theta_0}P^m_n(\cos\theta_0).
\end{align}
Finally there are the results from solving the integrals,
\begin{align}
  \label{eq:integrals}
  I^\pm(\rho,\phi) = \pi 
       \left( e^{i(m-1)\phi} J_{1-m}(k \sin \theta_0 \rho) \,\, \pm \right.\\
         \left. e^{i(m+1)\phi} J_{-1-m}(k \sin \theta_0 \rho) \right).
\end{align}
The BSCs for the Bessel beam are still complicated, however this form is
much more useful than the integral in Eq.~(\ref{eq:pmnintegral}). All one has 
to do to calculate $p_{mn}$ is to evaluate Bessel functions and associated 
Legendre polynomials.

The next step is to determine the scattered field from the BSCs. The BSCs 
depend on position because they refer to a spherical polar cooridinate system
centered on the particle. This allows us to easily match the boundary 
conditions which determines the coefficients of the scattered field
\cite{barton_internal_1988}. One notational issue is Barton uses a different
convention for the BSCs but they are related according to,
\begin{equation}
  \label{eq:notation}
  A_{mn} = \frac{ip_{mn}}{2\pi k a}, \,\,\,\, 
  B_{mn} = \frac{n_{ext}q_{mn}}{2\pi k a}.
\end{equation}
The coefficients of the scattered field are,
\begin{align}
  \label{eq:scatteredfieldcoeff}
  a_{mn} &= \frac{\psi_n^\prime(k_{int}a)\psi_n(k_{ext}a)-
                   \bar{n}\psi_n(k_{int}a)\psi_n^\prime(k_{ext}a)}
                {\bar{n}\psi_n(k_{int}a)\xi_n^{(1)\prime}(k_{ext}a)-
                   \psi_n^\prime(k_{int}a)\xi_n^{(1)}(k_{ext}a)} A_{mn} ,\\
  b_{mn} &= \frac{\bar{n}\psi_n^\prime(k_{int}a)\psi(k_{ext}a)-
                   \psi_n(k_{int}a)\psi_n^\prime(k_{ext}a)}
                {\psi_n(k_{int}a)\xi_n^{(1)\prime}(k_{ext}a)-
                   \bar{n}\psi_n^\prime(k_{int}a)\xi_n^{(1)}(k_{ext}a)} B_{mn}, 
\end{align}
Where $\xi_n^{(1)}= \psi_n - i \chi_n$ and $\psi_n$ and $\chi_n$ are the 
Ricatti-Bessel functions. Additionally, $k_{int}$ is the wavenumber 
inside the sphere, $k_{ext}$ is the wavenumber outside the sphere, and
$a$ is the radius of the spherical particle. Finally the complex index
of refraction is given by $\bar{n}=\sqrt{\bar{\epsilon}_{int}/\epsilon_{ext}}$  
where $\bar{\epsilon}_{int} = \epsilon_{int} + i4\pi \sigma/\omega$ and 
$\sigma$ is the electrical conductivity of the particle. This can 
be simplified using the $a_n$ and $b_n$ in Bohren and Huffman, 
\begin{align}
  \label{eq:scatteredfieldcoeff}
  a_{mn} &= -a_n A_{mn} ,\\
  b_{mn} &= -b_n B_{mn}, 
\end{align}
(CHECK).

Given these BSCs for the incident and the scattered fields
we can now calculate the force on a spherical particle
using the formalism derived by Barton et. al. \cite{barton_theoretical_1989}.
Integrating the Maxwell's stress tensor,$\mathbf{T}$ ,
 over a surface enclosing the particle
determines the optical force,
\begin{equation}
  \label{eq:maxwellforce}
  <\mathbf{F} \pos> = < \oint_S \hat{n} \cdot \mathbf{T} dS >.
\end{equation}
Barton substitutes the series expression for the fields,
 Eq.~(\ref{eq:expansion}), into the surface integral and then evaluates
in the limit large radius. It's possible then to directly integrate
and express the force on the particle in terms of the BSCs of the 
incident, $A_{mn}$ and $B_{mn}$ and scattered fields,$a_{mn}$ and $b_{mn}$. The
 expression is huge and nasty:
\begin{widetext}
\begin{align}
  \label{eq:forcexy}
  \frac{<F_x>+i<F_y>}{a^2 E_0^2} = 
     & \frac{\alpha^2}{16 \pi} i \sum_{n=1}^{\infty}\sum_{m=-n}^{n}
        \left[\sqrt{\frac{(n+m+2)(n+m+1)}{(2n+1)(2n+3)}}n(n+2) \times \right.\\
           & \left.(2 \epsilon_{ext} a_{mn}a_{m+1,n+1}^*+
              \epsilon_{ext} a_{mn}A_{m+1,n+1}^*+
              \epsilon_{ext} A_{mn}a_{m+1,n+1}^*+
              2 b_{mn}b_{m+1,n+1}^*+
              b_{mn}B_{m+1,n+1}^*+ \right. \\
              & \left. B_{mn}b_{m+1,n+1}^* ) +
           \sqrt{\frac{(n-m+1)(n-m+2)}{(2n+1)(2n+3)}}n(n+2) (
              2 \epsilon_{ext} a_{m-1,n+1}a_{mn}^*+
              \epsilon_{ext} a_{m-1,n+1}A_{mn}^*+ \right. \\
              & \left. \epsilon_{ext} A_{m-1,n+1}a_{mn}^*+
              2 b_{m-1,n+1}b_{mn}^*+
              b_{m-1,n+1}B_{mn}^*+
              B_{m-1,n+1}b_{mn}^*) \right. \\
           & \left. -\sqrt{(n+m+1)(n-m)}\sqrt{\epsilon_{ext})} (
              -2 a_{mn}b_{n,m+1}^*+
              2 b_{mn}a_{n,m+1}^*+
              -a_{mn}B_{n,m+1}^*+\right. \\
              & \left. b_{mn}A_{n,m+1}^*+
              B_{mn}a_{n,m+1}^*+
              -A_{mn}b_{n,m+1}^*)\right]
\end{align}
and,
\begin{align}
  \label{eq:forcez}
  \frac{<F_z>}{a^2 E_0^2} =
     & -\frac{\alpha^2}{8 \pi} \sum_{n=1}^{\infty}\sum_{m=-n}^{n} \Im 
        \left\{ \sqrt{\frac{(n-m+1)(n+m+1)}{(2n+1)(2n+3)}}n(n+2)\times
            \right. \\
            & \left. (2 \epsilon_{ext} a_{m,n+1}a_{mn}^*+ 
               \epsilon_{ext} a_{m,n+1}A_{mn}^*+
               \epsilon_{ext} A_{m,n+1}a_{mn}^*+
               2 b_{m,n+1}b_{mn}^*+ 
               b_{m,n+1}B_{mn}^*+ \right.\\
               & \left. B_{m,n+1}b_{mn}^*)+
            \sqrt{\epsilon_{ext}}m(2a_{mn}b_{mn}^*+
               a_{mn}B_{mn}^*+
               A_{mn}b_{mn}^*)\right\}.
\end{align}
\end{widetext}
There are also similar expressions in Farsund and Felderhof
 \cite{farsund_force_1996} and in Mazolli et. al. \cite{mazolli_theory_2003}.
So we should be able to catch any mistakes in these equations.


%\bibliography{abbreviations,grier,tweezer,dgg}
%\input{soconveyor1.bbl}
\bibliography{tractorrangeLibrary}
%\bibliographystyle{publist}
\bibliographystyle{chicago}

\end{document} 
